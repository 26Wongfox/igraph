\documentclass[twoside]{book}% Specify document type
\usepackage{graphicx}           % Load macros for PostScript figures
\usepackage{url}
\input ICCSsty.tex              % Load style for Complex Systems '97

%---------------------------------------------------------
% Give data to appear in the title section

 \shtitle{The igraph software package for complex network research}
 \title{The igraph software package for complex network research}
 \author{%
   G\'abor Cs\'ardi\affil{Center for Complex Systems Studies,
     Kalamazoo College, Kalamazoo, MI, USA \\
     and \\
     Department of Biophysics, KFKI Research Institute for Particle
     and Nuclear Physics of the Hungarian Academy of Sciences,
     Budapest, Hungary\\csardi@kzoo.edu} 
   
   Tam\'as Nepusz\affil{Department of Biophysics, KFKI Research
     Institute for Particle and Nuclear Physics of the Hungarian
     Academy of Sciences, Budapest, Hungary\\ntamas@rmki.kfki.hu}
 }
\date{May 10, 2006}
\abstract{The igraph software package provides handy tools for
  researchers in network science. It is an open source portable
  library capable of handling huge graphs with millions of vertices
  and edges and it is also suitable to grid computing. It contains
  routines for creating, manipulating and visualizing networks,
  calculating various structural properties, importing from and
  exporting to various file formats and many more. Via its interfaces
  to high-level languages like GNU R and Python it supports rapid
  development and fast prototyping.
}

%---------------------------------------------------------
% Here is the document

\begin{document}           % Matched by \end{document}
\maketitle                 % Typeset the title section
                           
\section{Introduction}

This paper does not present results of scientific research, but
introduces a software package which gives handy tools into the hands
of researchers doing network science. The authors strongly believe
that the tools scientists use are important because that they
can increase productivity by several factors and thereby enhance
scientific progress. 

\subsection{Why another network analysis package?}

The igraph library was developed because of the lack of network
analysis software which (1)~can handle large graphs efficiently,
(2)~is embeddable into a higher level program or programming language
(like Python, Perl or GNU R) and (3) can be used both interactively
and non-interactively.

The capability of handling large graphs was important because the
authors were confronted with graphs with millions of vertices and edges. 

Embedding igraph into Python or GNU R creates a very productive
research environment, well suitable for rapid development. All the
expressing power of GNU R (or other higher level languange) is readily
available in a convenient integrated environment for generating,
manipualing and measuring graphs, and evaluating these measurements.

Interactive means of software usage is nowadays considered as superior
to non-interactive interfaces, which is very true for most
cases. Dealing with large graphs can be different though, if it takes
three month to calculate the diameter of a graph, nobody wants that to
be interactive.

\section{Features}

In the addition to the three goal features in the previous section
others showed up as a side-effect. Let us discuss all features here.

\paragraph{Open source.} Igraph is open source, it is free for
non-commercial or commercial use and distributed according to the GNU
General Public License. Beeing open source means that in addition to
the binary format of the program the user can
always get the source code format enabling additions and corrections.
This is a very important feature for the users. How many times did you
find a piece of software which was perfect for your (or your
company's) needs except for some small tidbits. With open source
software you can add or correct these deficiencies or hire somebody to
this for your, with closed source software this is impossible.

\paragraph{Efficient implementation.} Igraph uses space and time
efficient data structures and implements the current state-of-the-art
algorithms. All igraph functions are carefully profiled to create the
fastest implementation possible.

\paragraph{Portability.} The library is written in ANSI~C it is thus
portable to most platforms, it is tested on different Linux flavours,
Mac OSX, MS Windows, and Sun OS. The R and Python interfaces are also
portable to many architectures.

\paragraph{Layered architecture.} The igraph library has a layered
architecture, the three layers are connected through well defined
interfaces. Each layer can be replaced with an alternate
implementation without changing the other components. See the details
in~\sect{arch}.

\paragraph{Open, embeddable system.} The core igraph library is an
open system, it can be embedded into higher level languages or
programs, the current distribution contains interfaces to two high
level languages: GNU R and Python.

\paragraph{High level operations.} The higher level interfaces provide
abstract, high level operations and data types. These support rapid
program development, see~\sect{fast} for an example.

\paragraph{Documentation.} The C library is very well documented, the
documentation is available is various forms supporting both online
browsing and printing. For each function its time requirements are
documented.

\paragraph{Drawbacks.} The library lacks functionality in some areas
compared to other network analysis packages. One such area is graph
visualization, another one is various social network analysis methods
like blockmodeling, p$^{*}$ methods, etc. Note that this piece of
software is heavily under development, so expect much more
functionality in the near future. Igraph also does not have a
graphical user interface.

\section{Example applications}

\subsection{Grid computing}

\subsection{Fast prototyping}\sectlabel{fast}

\section{The igraph architecture}\sectlabel{arch}

\section{Current functionality}

Please note that new functionality is added to the library almost each
day, so check the igraph homepage at
\url{http://cneurocvs.rmki.kfki.hu/igraph} if you cannot see here the
algorithms or measures you're looking for. 

\paragraph{Graph generation} Igraph can generate various regular and
random graphs: $\bullet$ regular structures: star, ring and full
graphs, circular and non-circular lattices with any number of
dimensions, regular trees $\bullet$ graphs based on Barab\'asi's
preferential attachment model \cite{barabasi99a}, also with nonlinear
attachment exponent and various variation $\bullet$ Random
(Erd\H{o}s-R\'enyi) graphs, both $G(n,p)$ and $G(n,m)$ types, directed
and undirected ones $\bullet$ graphs having a given degree sequence,
directed or undirected ones $\bullet$ growing random graphs, also for
modeling citation networks $\bullet$ growing random graphs where the
connection probability depends on some vertex properties $\bullet$
graphs from the Graph Atlas $\bullet$ all non-isomorphic graphs of a
given size.

\paragraph{Centrality measures} The following centraility measures can
be calculated: $\bullet$ degree $\bullet$ closeness $\bullet$
vertex and edge betweenness $\bullet$ eigenvector centrality $\bullet$
page rank.

\paragraph{Path length based properties} One or all shortest paths
between vertices can be calculated, and also based on this the
diameter and the average path length of the graph.

\paragraph{Graph components} Weakly and strongly connected components
can be calculated, and also the minimum spanning forest of a graph.

\paragraph{Graph motifs} Graph motifs of three or four components can
be calculated, both undirected and directed motifs.

\paragraph{Vertex and edge sets} Igraph provides a simple way to
manipulate subsets of vertices and/or edges of a graph,
see~\sect{fast} for an example. 

\paragraph{Vertex and edge attributes} Numeric or non-numeric
attributes can be assigned to the vertices and edges of a graph, and
queried and set by using a simple notation, see~\sect{fast}. 

\paragraph{File formats} Igraph can read and write various simple edge
list files and also Pajek and GraphML files.

\paragraph{Graph layouts} The following layout generators are part of
igraph: $\bullet$ simple circle and sphere layouts, random layouts
$\bullet$ Fruchterman-Reingold layout, 2D and 3D $\bullet$
Kamada-Kawai layout, 2D and 3D $bullet$ spring embedder layout
$\bullet$ LGL layout generator for large graphs $\bullet$ Grid-based
Fruchterman-Reingold layout for large graphs $\bullet$
Reingold-Tilford layout for trees.

\bibliography{net}

\end{document}
