\documentclass[12pt,letterpaper]{article}

\usepackage{times}
\usepackage{fullpage}

\setlength{\parindent}{0pt}
\setlength{\parskip}{10pt}

\begin{document}
\author{\bf G\'abor Cs\'ardi$^{1,2}$ and Tam\'as Nepusz$^{1,3}$}
\title{\Large\bf The \texttt{igraph} software package for complex
  network research} 
\date{\bf\normalsize\texttt{\{csardi,ntamas\}@rmki.kfki.hu}\\[15pt]
  $^1$ Department of Biophysics,\\ Research Institute for Nuclear and
  Particle Phyisics \\ of the Hungarian Academy of Sciences, \\
  29-33 Konkoly-Thege \'ut, Budapest 1121\\[10pt]
  $^2$ Center for Complex Systems Studies, Kalamazoo College, \\
  1200 Academy st, Kalamazoo, MI 49006 \\[10pt]
  $^3$ Department of Measurement and Information Systems, \\ Budapest
  University of Technology and Economics, \\
  3-9 M\H{u}egyetem rkp, Budapest 1111
}
\maketitle

\pagestyle{empty}\thispagestyle{empty}
\enlargethispage{5cm}
\centerline{\bf\large Abstract}

\noindent 
This presentation does not cover results of scientific research, but
introduces a software package which gives handy tools into the hands
of researchers doing network science. The authors strongly believe
that the tools scientists use are important because they
can increase productivity by several factors and thereby enhance
scientific progress. 

The \texttt{igraph} library was developed because of the lack of network
analysis software which (1)~can handle large graphs efficiently,
(2)~can be embedded into a higher level program or programming language
(like Python, Perl or GNU R) and (3) can be used both interactively
and non-interactively.

The capability of handling large graphs was important because the
authors were confronted with graphs with millions of vertices and edges. 

Embedding \texttt{igraph} into Python or GNU R creates a very productive
research environment, well suited for rapid development. All the
expressing power of GNU R (or other higher level languange) is readily
available in a convenient integrated environment for generating,
manipulating and measuring graphs, and evaluating these measurements.

Interactive means of software usage is nowadays considered as superior
to non-interactive interfaces, which is very true for most
cases. Dealing with large graphs can be different though -- if it takes
three months to calculate the diameter of a graph, nobody wants that to
be interactive.

In this presentation we will show the key features of the
\texttt{igraph} software package and various demonstration
tasks involving generating, analyzing and visualizing large graphs.

\vspace*{10pt}
\noindent
\textbf{Key Words:} network analysis software, large networks

\end{document}
